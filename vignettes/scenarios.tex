% Options for packages loaded elsewhere
\PassOptionsToPackage{unicode}{hyperref}
\PassOptionsToPackage{hyphens}{url}
%
\documentclass[
]{article}
\usepackage{amsmath,amssymb}
\usepackage{lmodern}
\usepackage{iftex}
\ifPDFTeX
  \usepackage[T1]{fontenc}
  \usepackage[utf8]{inputenc}
  \usepackage{textcomp} % provide euro and other symbols
\else % if luatex or xetex
  \usepackage{unicode-math}
  \defaultfontfeatures{Scale=MatchLowercase}
  \defaultfontfeatures[\rmfamily]{Ligatures=TeX,Scale=1}
\fi
% Use upquote if available, for straight quotes in verbatim environments
\IfFileExists{upquote.sty}{\usepackage{upquote}}{}
\IfFileExists{microtype.sty}{% use microtype if available
  \usepackage[]{microtype}
  \UseMicrotypeSet[protrusion]{basicmath} % disable protrusion for tt fonts
}{}
\makeatletter
\@ifundefined{KOMAClassName}{% if non-KOMA class
  \IfFileExists{parskip.sty}{%
    \usepackage{parskip}
  }{% else
    \setlength{\parindent}{0pt}
    \setlength{\parskip}{6pt plus 2pt minus 1pt}}
}{% if KOMA class
  \KOMAoptions{parskip=half}}
\makeatother
\usepackage{xcolor}
\usepackage[margin=1in]{geometry}
\usepackage{color}
\usepackage{fancyvrb}
\newcommand{\VerbBar}{|}
\newcommand{\VERB}{\Verb[commandchars=\\\{\}]}
\DefineVerbatimEnvironment{Highlighting}{Verbatim}{commandchars=\\\{\}}
% Add ',fontsize=\small' for more characters per line
\usepackage{framed}
\definecolor{shadecolor}{RGB}{248,248,248}
\newenvironment{Shaded}{\begin{snugshade}}{\end{snugshade}}
\newcommand{\AlertTok}[1]{\textcolor[rgb]{0.94,0.16,0.16}{#1}}
\newcommand{\AnnotationTok}[1]{\textcolor[rgb]{0.56,0.35,0.01}{\textbf{\textit{#1}}}}
\newcommand{\AttributeTok}[1]{\textcolor[rgb]{0.77,0.63,0.00}{#1}}
\newcommand{\BaseNTok}[1]{\textcolor[rgb]{0.00,0.00,0.81}{#1}}
\newcommand{\BuiltInTok}[1]{#1}
\newcommand{\CharTok}[1]{\textcolor[rgb]{0.31,0.60,0.02}{#1}}
\newcommand{\CommentTok}[1]{\textcolor[rgb]{0.56,0.35,0.01}{\textit{#1}}}
\newcommand{\CommentVarTok}[1]{\textcolor[rgb]{0.56,0.35,0.01}{\textbf{\textit{#1}}}}
\newcommand{\ConstantTok}[1]{\textcolor[rgb]{0.00,0.00,0.00}{#1}}
\newcommand{\ControlFlowTok}[1]{\textcolor[rgb]{0.13,0.29,0.53}{\textbf{#1}}}
\newcommand{\DataTypeTok}[1]{\textcolor[rgb]{0.13,0.29,0.53}{#1}}
\newcommand{\DecValTok}[1]{\textcolor[rgb]{0.00,0.00,0.81}{#1}}
\newcommand{\DocumentationTok}[1]{\textcolor[rgb]{0.56,0.35,0.01}{\textbf{\textit{#1}}}}
\newcommand{\ErrorTok}[1]{\textcolor[rgb]{0.64,0.00,0.00}{\textbf{#1}}}
\newcommand{\ExtensionTok}[1]{#1}
\newcommand{\FloatTok}[1]{\textcolor[rgb]{0.00,0.00,0.81}{#1}}
\newcommand{\FunctionTok}[1]{\textcolor[rgb]{0.00,0.00,0.00}{#1}}
\newcommand{\ImportTok}[1]{#1}
\newcommand{\InformationTok}[1]{\textcolor[rgb]{0.56,0.35,0.01}{\textbf{\textit{#1}}}}
\newcommand{\KeywordTok}[1]{\textcolor[rgb]{0.13,0.29,0.53}{\textbf{#1}}}
\newcommand{\NormalTok}[1]{#1}
\newcommand{\OperatorTok}[1]{\textcolor[rgb]{0.81,0.36,0.00}{\textbf{#1}}}
\newcommand{\OtherTok}[1]{\textcolor[rgb]{0.56,0.35,0.01}{#1}}
\newcommand{\PreprocessorTok}[1]{\textcolor[rgb]{0.56,0.35,0.01}{\textit{#1}}}
\newcommand{\RegionMarkerTok}[1]{#1}
\newcommand{\SpecialCharTok}[1]{\textcolor[rgb]{0.00,0.00,0.00}{#1}}
\newcommand{\SpecialStringTok}[1]{\textcolor[rgb]{0.31,0.60,0.02}{#1}}
\newcommand{\StringTok}[1]{\textcolor[rgb]{0.31,0.60,0.02}{#1}}
\newcommand{\VariableTok}[1]{\textcolor[rgb]{0.00,0.00,0.00}{#1}}
\newcommand{\VerbatimStringTok}[1]{\textcolor[rgb]{0.31,0.60,0.02}{#1}}
\newcommand{\WarningTok}[1]{\textcolor[rgb]{0.56,0.35,0.01}{\textbf{\textit{#1}}}}
\usepackage{graphicx}
\makeatletter
\def\maxwidth{\ifdim\Gin@nat@width>\linewidth\linewidth\else\Gin@nat@width\fi}
\def\maxheight{\ifdim\Gin@nat@height>\textheight\textheight\else\Gin@nat@height\fi}
\makeatother
% Scale images if necessary, so that they will not overflow the page
% margins by default, and it is still possible to overwrite the defaults
% using explicit options in \includegraphics[width, height, ...]{}
\setkeys{Gin}{width=\maxwidth,height=\maxheight,keepaspectratio}
% Set default figure placement to htbp
\makeatletter
\def\fps@figure{htbp}
\makeatother
\setlength{\emergencystretch}{3em} % prevent overfull lines
\providecommand{\tightlist}{%
  \setlength{\itemsep}{0pt}\setlength{\parskip}{0pt}}
\setcounter{secnumdepth}{-\maxdimen} % remove section numbering
\ifLuaTeX
  \usepackage{selnolig}  % disable illegal ligatures
\fi
\IfFileExists{bookmark.sty}{\usepackage{bookmark}}{\usepackage{hyperref}}
\IfFileExists{xurl.sty}{\usepackage{xurl}}{} % add URL line breaks if available
\urlstyle{same} % disable monospaced font for URLs
\hypersetup{
  pdftitle={Summary of data generation scenarios},
  hidelinks,
  pdfcreator={LaTeX via pandoc}}

\title{Summary of data generation scenarios}
\author{}
\date{\vspace{-2.5em}}

\begin{document}
\maketitle

\begin{Shaded}
\begin{Highlighting}[]
\FunctionTok{library}\NormalTok{(dpsurrogate)}
\FunctionTok{library}\NormalTok{(splines)}
\FunctionTok{library}\NormalTok{(knitr)}

\NormalTok{summary\_scenario }\OtherTok{\textless{}{-}} \ControlFlowTok{function}\NormalTok{(idat) \{}
  \FunctionTok{par}\NormalTok{(}\AttributeTok{mfrow =} \FunctionTok{c}\NormalTok{(}\DecValTok{2}\NormalTok{, }\DecValTok{2}\NormalTok{))}
  \FunctionTok{plot}\NormalTok{(yeff }\SpecialCharTok{\textasciitilde{}}\NormalTok{ seff, }\AttributeTok{data =}\NormalTok{ idat}\SpecialCharTok{$}\NormalTok{rdat)}
  \FunctionTok{plot}\NormalTok{(yeff }\SpecialCharTok{\textasciitilde{}}\NormalTok{ trtZ, }\AttributeTok{data =}\NormalTok{ idat}\SpecialCharTok{$}\NormalTok{rdat)}
  
  \DocumentationTok{\#\# partial residuals}
  
\NormalTok{  fit.z }\OtherTok{\textless{}{-}} \FunctionTok{lm}\NormalTok{(yeff }\SpecialCharTok{\textasciitilde{}} \FunctionTok{bs}\NormalTok{(trtZ, }\AttributeTok{df =} \DecValTok{3}\NormalTok{), }\AttributeTok{data =}\NormalTok{ idat}\SpecialCharTok{$}\NormalTok{rdat)}
\NormalTok{  fit.s }\OtherTok{\textless{}{-}} \FunctionTok{lm}\NormalTok{(yeff }\SpecialCharTok{\textasciitilde{}} \FunctionTok{bs}\NormalTok{(seff, }\AttributeTok{df =} \DecValTok{3}\NormalTok{), }\AttributeTok{data =}\NormalTok{ idat}\SpecialCharTok{$}\NormalTok{rdat)}
  
  \FunctionTok{plot}\NormalTok{(}\FunctionTok{resid}\NormalTok{(fit.z) }\SpecialCharTok{\textasciitilde{}}\NormalTok{ idat}\SpecialCharTok{$}\NormalTok{rdat}\SpecialCharTok{$}\NormalTok{seff, }\AttributeTok{ylab =} \StringTok{"p resid | Z"}\NormalTok{, }\AttributeTok{xlab =} \StringTok{"seff"}\NormalTok{)}
  \FunctionTok{plot}\NormalTok{(}\FunctionTok{resid}\NormalTok{(fit.s) }\SpecialCharTok{\textasciitilde{}}\NormalTok{ idat}\SpecialCharTok{$}\NormalTok{rdat}\SpecialCharTok{$}\NormalTok{trtZ, }\AttributeTok{ylab =} \StringTok{"p resid | seff"}\NormalTok{, }\AttributeTok{xlab =} \StringTok{"trtZ"}\NormalTok{)}
  
  \FunctionTok{summary}\NormalTok{(}\FunctionTok{lm}\NormalTok{(yeff }\SpecialCharTok{\textasciitilde{}}\NormalTok{ seff }\SpecialCharTok{+}\NormalTok{ trtZ, }\AttributeTok{data =}\NormalTok{ idat}\SpecialCharTok{$}\NormalTok{rdat))}
  
\NormalTok{\}}
\end{Highlighting}
\end{Shaded}

We start by generating \(\nu_j, Z_j, U_j\) for \(j = 1, \ldots, 64\) all
from independent standard normal distributions. Recall that \(\nu_j\)
represents the treatment effect on the potential surrogate in group
\(i\), and \(Z_j\) is an observed treatment-level covariate, and \(U_i\)
is an unobserved treatment-level covariate. We consider the following
scenarios for generation of \(\mu_j\), the treatment effect on the
clinical outcome.

\begin{itemize}
\tightlist
\item
  Nonlinear (nonlinear):
  \(\mu_j = -1 + f(\nu_j) + c_z |Z_j| + c_u U_j\), where \(f\) is a
  linear spline basis with knots at \(-1, 0, 1\) and coefficients chosen
  so that the trend is monotonically increasing.
\item
  Nonlinear skew (nonlinearskew), same as above but \(\nu_j\) are
  sampled from a skew-normal distribution.
\item
  Linear (linear): \(\mu_j = -1 + 1* \nu_j + c_z |Z_j| + c_u U_j\).
\item
  Simple (simple): \(\mu_j = -1 + 1* \nu_j + c_z Z_j + c_u U_j\).
\item
  Simple strong (simplestrong):
  \(\mu_j = -1 + 2* \nu_j + c_z Z_j + c_u U_j\).
\item
  Null (null): \(\mu_j = -1 + c_z |Z_j| + c_u U\).
\item
  Interaction (inter): \(\mu_j = 1\{Z_j < 0\} * \nu_j + c_u U_j\), where
  \(1\{\cdot\}\) is the - r function.
\item
  Hidden interaction (interhide):
  \(\mu_j = 1\{U_j < 0\} * \nu_j + c_z Z_j\).
\item
  Surrogate value for only 1 treatment (onetrt):
  \(\mu_j = 1\{j\in R_a\} * \nu_j + c_z Z_j + c_u U_j\) where \(R_a\) is
  the set of indices such that the active treatment is treatment A.
\item
  Surrogate value for 2 treatments (twotrt):
  \(\mu_j = 1\{j\in (R_a, R_b)\} * (\nu_j - \min_j(\nu_j)) + c_z Z_j + c_u U_j\)
  where \(R_a, R_b\) are the sets of indices such that the active
  treatment is treatment A and B, respectively. Additionally, the mean
  of \(Z_j\) is changed to vary based on the treatment group (with
  different means in each of the 4 groups).
\item
  Surrogate value for multiple biomarkers (manybiom): The biomarker
  groups are grouped into 3 categories \(B^*_j \in \{1, 2, 3\}\) of size
  20, 28, 16, respectively. Then \(Z_j \sim N(B^*_j - 1, 0.5^2)\) and
  \(\mu_j = 0.25 \cdot (B^*_j - 3) \cdot (\nu_j - \min_j(\nu_j)) + c_z Z_j + c_u U_j\).
  In this setting, the three biomarker groups have differential
  surrogate value none, moderate, and strong.
\end{itemize}

\begin{Shaded}
\begin{Highlighting}[]


\NormalTok{settings }\OtherTok{\textless{}{-}} \FunctionTok{rbind}\NormalTok{(}\FunctionTok{data.frame}\NormalTok{(}\AttributeTok{scen =} \FunctionTok{c}\NormalTok{(}\StringTok{"nonlinear"}\NormalTok{, }\StringTok{"nonlinearskew"}\NormalTok{, }\StringTok{"linear"}\NormalTok{, }\StringTok{"simple"}\NormalTok{, }\StringTok{"simplestrong"}\NormalTok{,}
                                      \StringTok{"null"}\NormalTok{, }\StringTok{"inter"}\NormalTok{, }\StringTok{"interhide"}\NormalTok{,}
                    \StringTok{"onetrt"}\NormalTok{, }\StringTok{"twotrt"}\NormalTok{, }\StringTok{"manybiom"}\NormalTok{),}
           \AttributeTok{num =} \DecValTok{99}\NormalTok{, }\AttributeTok{p1 =} \DecValTok{0}\NormalTok{, }\AttributeTok{p2 =} \DecValTok{0}\NormalTok{),}
      \FunctionTok{data.frame}\NormalTok{(}\AttributeTok{scen =} \FunctionTok{c}\NormalTok{(}\StringTok{"nonlinear"}\NormalTok{, }\StringTok{"linear"}\NormalTok{, }\StringTok{"simple"}\NormalTok{),}
                 \AttributeTok{num =} \DecValTok{99}\NormalTok{, }\AttributeTok{p1 =} \FloatTok{0.3}\NormalTok{, }\AttributeTok{p2 =} \DecValTok{0}\NormalTok{),}
      \FunctionTok{data.frame}\NormalTok{(}\AttributeTok{scen =} \FunctionTok{c}\NormalTok{(}\StringTok{"nonlinear"}\NormalTok{, }\StringTok{"linear"}\NormalTok{, }\StringTok{"simple"}\NormalTok{, }\StringTok{"manybiom"}\NormalTok{),}
                 \AttributeTok{num =} \DecValTok{99}\NormalTok{, }\AttributeTok{p1 =} \FloatTok{0.3}\NormalTok{, }\AttributeTok{p2 =} \FloatTok{0.3}\NormalTok{),}
      \FunctionTok{data.frame}\NormalTok{(}\AttributeTok{scen =} \FunctionTok{c}\NormalTok{(}\StringTok{"inter"}\NormalTok{, }\StringTok{"onetrt"}\NormalTok{),}
                 \AttributeTok{num =} \DecValTok{99}\NormalTok{, }\AttributeTok{p1 =} \DecValTok{0}\NormalTok{, }\AttributeTok{p2 =} \FloatTok{0.3}\NormalTok{),}
      \FunctionTok{data.frame}\NormalTok{(}\AttributeTok{scen =} \FunctionTok{c}\NormalTok{(}\StringTok{"interhide"}\NormalTok{, }\StringTok{"twotrt"}\NormalTok{, }\StringTok{"manybiom"}\NormalTok{),}
                 \AttributeTok{num =} \DecValTok{99}\NormalTok{, }\AttributeTok{p1 =} \FloatTok{0.3}\NormalTok{, }\AttributeTok{p2 =} \DecValTok{0}\NormalTok{))}

\NormalTok{settings}\SpecialCharTok{$}\NormalTok{fname }\OtherTok{\textless{}{-}} \FunctionTok{gsub}\NormalTok{(}\StringTok{"."}\NormalTok{, }\StringTok{""}\NormalTok{, }\FunctionTok{make.names}\NormalTok{(settings}\SpecialCharTok{$}\NormalTok{scen, }\AttributeTok{unique =} \ConstantTok{TRUE}\NormalTok{), }\AttributeTok{fixed =} \ConstantTok{TRUE}\NormalTok{)}


\ControlFlowTok{for}\NormalTok{(i }\ControlFlowTok{in} \DecValTok{1}\SpecialCharTok{:}\FunctionTok{nrow}\NormalTok{(settings))\{}
\NormalTok{  dscen }\OtherTok{\textless{}{-}} \ControlFlowTok{if}\NormalTok{(settings}\SpecialCharTok{$}\NormalTok{scen[i] }\SpecialCharTok{==} \StringTok{"nonlinearskew"}\NormalTok{) }\StringTok{"nonlinear"} \ControlFlowTok{else}\NormalTok{ settings}\SpecialCharTok{$}\NormalTok{scen[i]}
\NormalTok{  idat }\OtherTok{\textless{}{-}} \FunctionTok{generate\_data}\NormalTok{(}\AttributeTok{effect =}\NormalTok{ dscen, }\AttributeTok{Zeffect =}\NormalTok{ settings}\SpecialCharTok{$}\NormalTok{p1[i], }
                        \AttributeTok{Ueffect =}\NormalTok{ settings}\SpecialCharTok{$}\NormalTok{p2[i])}
\NormalTok{  header }\OtherTok{\textless{}{-}} \FunctionTok{sprintf}\NormalTok{(}\StringTok{"}\SpecialCharTok{\textbackslash{}\textbackslash{}}\StringTok{subsection\{Setting: \%s, $c\_z = \%.1f, c\_u = \%.1f$\} }\SpecialCharTok{\textbackslash{}n\textbackslash{}n}\StringTok{"}\NormalTok{, }
\NormalTok{              settings}\SpecialCharTok{$}\NormalTok{scen[i], settings}\SpecialCharTok{$}\NormalTok{p1[i], }
\NormalTok{              settings}\SpecialCharTok{$}\NormalTok{p2[i])}
  \FunctionTok{cat}\NormalTok{(header)}
  \FunctionTok{summary\_scenario}\NormalTok{(idat)}
  
\NormalTok{\}}
\CommentTok{\#\textgreater{} \textbackslash{}subsection\{Setting: nonlinear, $c\_z = 0.0, c\_u = 0.0$\}}
\end{Highlighting}
\end{Shaded}

\includegraphics{scenarios_files/figure-latex/inter-1.pdf}

\begin{verbatim}
#> \subsection{Setting: nonlinearskew, $c_z = 0.0, c_u = 0.0$}
\end{verbatim}

\includegraphics{scenarios_files/figure-latex/inter-2.pdf}

\begin{verbatim}
#> \subsection{Setting: linear, $c_z = 0.0, c_u = 0.0$}
#> Warning in summary.lm(lm(yeff ~ seff + trtZ, data = idat$rdat)): essentially perfect fit: summary may be
#> unreliable
\end{verbatim}

\includegraphics{scenarios_files/figure-latex/inter-3.pdf}

\begin{verbatim}
#> \subsection{Setting: simple, $c_z = 0.0, c_u = 0.0$}
#> Warning in summary.lm(lm(yeff ~ seff + trtZ, data = idat$rdat)): essentially perfect fit: summary may be
#> unreliable
\end{verbatim}

\includegraphics{scenarios_files/figure-latex/inter-4.pdf}

\begin{verbatim}
#> \subsection{Setting: simplestrong, $c_z = 0.0, c_u = 0.0$}
#> Warning in summary.lm(lm(yeff ~ seff + trtZ, data = idat$rdat)): essentially perfect fit: summary may be
#> unreliable
\end{verbatim}

\includegraphics{scenarios_files/figure-latex/inter-5.pdf}

\begin{verbatim}
#> \subsection{Setting: null, $c_z = 0.0, c_u = 0.0$}
\end{verbatim}

\includegraphics{scenarios_files/figure-latex/inter-6.pdf}

\begin{verbatim}
#> \subsection{Setting: inter, $c_z = 0.0, c_u = 0.0$}
\end{verbatim}

\includegraphics{scenarios_files/figure-latex/inter-7.pdf}

\begin{verbatim}
#> \subsection{Setting: interhide, $c_z = 0.0, c_u = 0.0$}
\end{verbatim}

\includegraphics{scenarios_files/figure-latex/inter-8.pdf}

\begin{verbatim}
#> \subsection{Setting: onetrt, $c_z = 0.0, c_u = 0.0$}
\end{verbatim}

\includegraphics{scenarios_files/figure-latex/inter-9.pdf}

\begin{verbatim}
#> \subsection{Setting: twotrt, $c_z = 0.0, c_u = 0.0$}
\end{verbatim}

\includegraphics{scenarios_files/figure-latex/inter-10.pdf}

\begin{verbatim}
#> \subsection{Setting: manybiom, $c_z = 0.0, c_u = 0.0$}
\end{verbatim}

\includegraphics{scenarios_files/figure-latex/inter-11.pdf}

\begin{verbatim}
#> \subsection{Setting: nonlinear, $c_z = 0.3, c_u = 0.0$}
\end{verbatim}

\includegraphics{scenarios_files/figure-latex/inter-12.pdf}

\begin{verbatim}
#> \subsection{Setting: linear, $c_z = 0.3, c_u = 0.0$}
\end{verbatim}

\includegraphics{scenarios_files/figure-latex/inter-13.pdf}

\begin{verbatim}
#> \subsection{Setting: simple, $c_z = 0.3, c_u = 0.0$}
#> Warning in summary.lm(lm(yeff ~ seff + trtZ, data = idat$rdat)): essentially perfect fit: summary may be
#> unreliable
\end{verbatim}

\includegraphics{scenarios_files/figure-latex/inter-14.pdf}

\begin{verbatim}
#> \subsection{Setting: nonlinear, $c_z = 0.3, c_u = 0.3$}
\end{verbatim}

\includegraphics{scenarios_files/figure-latex/inter-15.pdf}

\begin{verbatim}
#> \subsection{Setting: linear, $c_z = 0.3, c_u = 0.3$}
\end{verbatim}

\includegraphics{scenarios_files/figure-latex/inter-16.pdf}

\begin{verbatim}
#> \subsection{Setting: simple, $c_z = 0.3, c_u = 0.3$}
\end{verbatim}

\includegraphics{scenarios_files/figure-latex/inter-17.pdf}

\begin{verbatim}
#> \subsection{Setting: manybiom, $c_z = 0.3, c_u = 0.3$}
\end{verbatim}

\includegraphics{scenarios_files/figure-latex/inter-18.pdf}

\begin{verbatim}
#> \subsection{Setting: inter, $c_z = 0.0, c_u = 0.3$}
\end{verbatim}

\includegraphics{scenarios_files/figure-latex/inter-19.pdf}

\begin{verbatim}
#> \subsection{Setting: onetrt, $c_z = 0.0, c_u = 0.3$}
\end{verbatim}

\includegraphics{scenarios_files/figure-latex/inter-20.pdf}

\begin{verbatim}
#> \subsection{Setting: interhide, $c_z = 0.3, c_u = 0.0$}
\end{verbatim}

\includegraphics{scenarios_files/figure-latex/inter-21.pdf}

\begin{verbatim}
#> \subsection{Setting: twotrt, $c_z = 0.3, c_u = 0.0$}
\end{verbatim}

\includegraphics{scenarios_files/figure-latex/inter-22.pdf}

\begin{verbatim}
#> \subsection{Setting: manybiom, $c_z = 0.3, c_u = 0.0$}
\end{verbatim}

\includegraphics{scenarios_files/figure-latex/inter-23.pdf}

\end{document}
