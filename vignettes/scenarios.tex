% Options for packages loaded elsewhere
\PassOptionsToPackage{unicode}{hyperref}
\PassOptionsToPackage{hyphens}{url}
%
\documentclass[
]{article}
\title{Summary of data generation scenarios}
\author{}
\date{\vspace{-2.5em}}

\usepackage{amsmath,amssymb}
\usepackage{lmodern}
\usepackage{iftex}
\ifPDFTeX
  \usepackage[T1]{fontenc}
  \usepackage[utf8]{inputenc}
  \usepackage{textcomp} % provide euro and other symbols
\else % if luatex or xetex
  \usepackage{unicode-math}
  \defaultfontfeatures{Scale=MatchLowercase}
  \defaultfontfeatures[\rmfamily]{Ligatures=TeX,Scale=1}
\fi
% Use upquote if available, for straight quotes in verbatim environments
\IfFileExists{upquote.sty}{\usepackage{upquote}}{}
\IfFileExists{microtype.sty}{% use microtype if available
  \usepackage[]{microtype}
  \UseMicrotypeSet[protrusion]{basicmath} % disable protrusion for tt fonts
}{}
\makeatletter
\@ifundefined{KOMAClassName}{% if non-KOMA class
  \IfFileExists{parskip.sty}{%
    \usepackage{parskip}
  }{% else
    \setlength{\parindent}{0pt}
    \setlength{\parskip}{6pt plus 2pt minus 1pt}}
}{% if KOMA class
  \KOMAoptions{parskip=half}}
\makeatother
\usepackage{xcolor}
\IfFileExists{xurl.sty}{\usepackage{xurl}}{} % add URL line breaks if available
\IfFileExists{bookmark.sty}{\usepackage{bookmark}}{\usepackage{hyperref}}
\hypersetup{
  pdftitle={Summary of data generation scenarios},
  hidelinks,
  pdfcreator={LaTeX via pandoc}}
\urlstyle{same} % disable monospaced font for URLs
\usepackage[margin=1in]{geometry}
\usepackage{color}
\usepackage{fancyvrb}
\newcommand{\VerbBar}{|}
\newcommand{\VERB}{\Verb[commandchars=\\\{\}]}
\DefineVerbatimEnvironment{Highlighting}{Verbatim}{commandchars=\\\{\}}
% Add ',fontsize=\small' for more characters per line
\usepackage{framed}
\definecolor{shadecolor}{RGB}{248,248,248}
\newenvironment{Shaded}{\begin{snugshade}}{\end{snugshade}}
\newcommand{\AlertTok}[1]{\textcolor[rgb]{0.94,0.16,0.16}{#1}}
\newcommand{\AnnotationTok}[1]{\textcolor[rgb]{0.56,0.35,0.01}{\textbf{\textit{#1}}}}
\newcommand{\AttributeTok}[1]{\textcolor[rgb]{0.77,0.63,0.00}{#1}}
\newcommand{\BaseNTok}[1]{\textcolor[rgb]{0.00,0.00,0.81}{#1}}
\newcommand{\BuiltInTok}[1]{#1}
\newcommand{\CharTok}[1]{\textcolor[rgb]{0.31,0.60,0.02}{#1}}
\newcommand{\CommentTok}[1]{\textcolor[rgb]{0.56,0.35,0.01}{\textit{#1}}}
\newcommand{\CommentVarTok}[1]{\textcolor[rgb]{0.56,0.35,0.01}{\textbf{\textit{#1}}}}
\newcommand{\ConstantTok}[1]{\textcolor[rgb]{0.00,0.00,0.00}{#1}}
\newcommand{\ControlFlowTok}[1]{\textcolor[rgb]{0.13,0.29,0.53}{\textbf{#1}}}
\newcommand{\DataTypeTok}[1]{\textcolor[rgb]{0.13,0.29,0.53}{#1}}
\newcommand{\DecValTok}[1]{\textcolor[rgb]{0.00,0.00,0.81}{#1}}
\newcommand{\DocumentationTok}[1]{\textcolor[rgb]{0.56,0.35,0.01}{\textbf{\textit{#1}}}}
\newcommand{\ErrorTok}[1]{\textcolor[rgb]{0.64,0.00,0.00}{\textbf{#1}}}
\newcommand{\ExtensionTok}[1]{#1}
\newcommand{\FloatTok}[1]{\textcolor[rgb]{0.00,0.00,0.81}{#1}}
\newcommand{\FunctionTok}[1]{\textcolor[rgb]{0.00,0.00,0.00}{#1}}
\newcommand{\ImportTok}[1]{#1}
\newcommand{\InformationTok}[1]{\textcolor[rgb]{0.56,0.35,0.01}{\textbf{\textit{#1}}}}
\newcommand{\KeywordTok}[1]{\textcolor[rgb]{0.13,0.29,0.53}{\textbf{#1}}}
\newcommand{\NormalTok}[1]{#1}
\newcommand{\OperatorTok}[1]{\textcolor[rgb]{0.81,0.36,0.00}{\textbf{#1}}}
\newcommand{\OtherTok}[1]{\textcolor[rgb]{0.56,0.35,0.01}{#1}}
\newcommand{\PreprocessorTok}[1]{\textcolor[rgb]{0.56,0.35,0.01}{\textit{#1}}}
\newcommand{\RegionMarkerTok}[1]{#1}
\newcommand{\SpecialCharTok}[1]{\textcolor[rgb]{0.00,0.00,0.00}{#1}}
\newcommand{\SpecialStringTok}[1]{\textcolor[rgb]{0.31,0.60,0.02}{#1}}
\newcommand{\StringTok}[1]{\textcolor[rgb]{0.31,0.60,0.02}{#1}}
\newcommand{\VariableTok}[1]{\textcolor[rgb]{0.00,0.00,0.00}{#1}}
\newcommand{\VerbatimStringTok}[1]{\textcolor[rgb]{0.31,0.60,0.02}{#1}}
\newcommand{\WarningTok}[1]{\textcolor[rgb]{0.56,0.35,0.01}{\textbf{\textit{#1}}}}
\usepackage{graphicx}
\makeatletter
\def\maxwidth{\ifdim\Gin@nat@width>\linewidth\linewidth\else\Gin@nat@width\fi}
\def\maxheight{\ifdim\Gin@nat@height>\textheight\textheight\else\Gin@nat@height\fi}
\makeatother
% Scale images if necessary, so that they will not overflow the page
% margins by default, and it is still possible to overwrite the defaults
% using explicit options in \includegraphics[width, height, ...]{}
\setkeys{Gin}{width=\maxwidth,height=\maxheight,keepaspectratio}
% Set default figure placement to htbp
\makeatletter
\def\fps@figure{htbp}
\makeatother
\setlength{\emergencystretch}{3em} % prevent overfull lines
\providecommand{\tightlist}{%
  \setlength{\itemsep}{0pt}\setlength{\parskip}{0pt}}
\setcounter{secnumdepth}{-\maxdimen} % remove section numbering
\ifLuaTeX
  \usepackage{selnolig}  % disable illegal ligatures
\fi

\begin{document}
\maketitle

\begin{Shaded}
\begin{Highlighting}[]
\FunctionTok{library}\NormalTok{(dpsurrogate)}
\FunctionTok{library}\NormalTok{(splines)}

\NormalTok{summary\_scenario }\OtherTok{\textless{}{-}} \ControlFlowTok{function}\NormalTok{(idat) \{}
  \FunctionTok{par}\NormalTok{(}\AttributeTok{mfrow =} \FunctionTok{c}\NormalTok{(}\DecValTok{2}\NormalTok{, }\DecValTok{2}\NormalTok{))}
  \FunctionTok{plot}\NormalTok{(yeff }\SpecialCharTok{\textasciitilde{}}\NormalTok{ seff, }\AttributeTok{data =}\NormalTok{ idat}\SpecialCharTok{$}\NormalTok{rdat)}
  \FunctionTok{plot}\NormalTok{(yeff }\SpecialCharTok{\textasciitilde{}}\NormalTok{ trtZ, }\AttributeTok{data =}\NormalTok{ idat}\SpecialCharTok{$}\NormalTok{rdat)}
  
  \DocumentationTok{\#\# partial residuals}
  
\NormalTok{  fit.z }\OtherTok{\textless{}{-}} \FunctionTok{lm}\NormalTok{(yeff }\SpecialCharTok{\textasciitilde{}} \FunctionTok{bs}\NormalTok{(trtZ, }\AttributeTok{df =} \DecValTok{3}\NormalTok{), }\AttributeTok{data =}\NormalTok{ idat}\SpecialCharTok{$}\NormalTok{rdat)}
\NormalTok{  fit.s }\OtherTok{\textless{}{-}} \FunctionTok{lm}\NormalTok{(yeff }\SpecialCharTok{\textasciitilde{}} \FunctionTok{bs}\NormalTok{(seff, }\AttributeTok{df =} \DecValTok{3}\NormalTok{), }\AttributeTok{data =}\NormalTok{ idat}\SpecialCharTok{$}\NormalTok{rdat)}
  
  \FunctionTok{plot}\NormalTok{(}\FunctionTok{resid}\NormalTok{(fit.z) }\SpecialCharTok{\textasciitilde{}}\NormalTok{ idat}\SpecialCharTok{$}\NormalTok{rdat}\SpecialCharTok{$}\NormalTok{seff, }\AttributeTok{ylab =} \StringTok{"p resid | Z"}\NormalTok{, }\AttributeTok{xlab =} \StringTok{"seff"}\NormalTok{)}
  \FunctionTok{plot}\NormalTok{(}\FunctionTok{resid}\NormalTok{(fit.s) }\SpecialCharTok{\textasciitilde{}}\NormalTok{ idat}\SpecialCharTok{$}\NormalTok{rdat}\SpecialCharTok{$}\NormalTok{trtZ, }\AttributeTok{ylab =} \StringTok{"p resid | seff"}\NormalTok{, }\AttributeTok{xlab =} \StringTok{"trtZ"}\NormalTok{)}
  
  \FunctionTok{summary}\NormalTok{(}\FunctionTok{lm}\NormalTok{(yeff }\SpecialCharTok{\textasciitilde{}}\NormalTok{ seff }\SpecialCharTok{+}\NormalTok{ trtZ, }\AttributeTok{data =}\NormalTok{ idat}\SpecialCharTok{$}\NormalTok{rdat))}
  
\NormalTok{\}}
\end{Highlighting}
\end{Shaded}

We start by generating \(\nu_j, Z_j, U_j\) for \(j = 1, \ldots, 64\) all
from independent standard normal distributions. Recall that \(\nu_j\)
represents the treatment effect on the potential surrogate in group
\(i\), and \(Z_j\) is an observed treatment-level covariate, and \(U_i\)
is an unobserved treatment-level covariate. We consider the following
scenarios for generation of \(\mu_j\), the treatment effect on the
clinical outcome.

\hypertarget{nonlinear}{%
\subsection{Nonlinear}\label{nonlinear}}

\[\mu_j = 1 - \sum_{i = 1}^4 \beta_i B_i(\nu_j) - \gamma_1 Z_j - \gamma_2 U_j,\]

where \(B_i(\cdot), i = 1, \ldots, 4\) is a linear B-spline basis with
knots at \(-1, 0, 1\), \(\beta = (0, .6, 3, 5)\).

When \(\gamma_1 = \gamma_2 = 0\), we have the following relationship:

\begin{Shaded}
\begin{Highlighting}[]
\NormalTok{idat }\OtherTok{\textless{}{-}} \FunctionTok{generate\_data}\NormalTok{(}\AttributeTok{effect =} \StringTok{"nonlinear"}\NormalTok{, }\AttributeTok{Zeffect =} \ConstantTok{FALSE}\NormalTok{, }\AttributeTok{Ueffect =} \ConstantTok{FALSE}\NormalTok{)}
\FunctionTok{summary\_scenario}\NormalTok{(idat)}
\end{Highlighting}
\end{Shaded}

\includegraphics{C:/Users/micsac/OneDrive - KI.SE/Methods/Bayesian DPP Surrogate/dpsurrogate/vignettes/scenarios_files/figure-latex/nonlin00-1.pdf}

\begin{verbatim}
#> 
#> Call:
#> lm(formula = yeff ~ seff + trtZ, data = idat$rdat)
#> 
#> Residuals:
#>      Min       1Q   Median       3Q      Max 
#> -1.36355 -0.30772  0.00826  0.35606  0.58445 
#> 
#> Coefficients:
#>             Estimate Std. Error t value Pr(>|t|)    
#> (Intercept) -0.18034    0.05395  -3.343  0.00142 ** 
#> seff         1.20332    0.06061  19.852  < 2e-16 ***
#> trtZ         0.05891    0.05352   1.101  0.27532    
#> ---
#> Signif. codes:  0 '***' 0.001 '**' 0.01 '*' 0.05 '.' 0.1 ' ' 1
#> 
#> Residual standard error: 0.4257 on 61 degrees of freedom
#> Multiple R-squared:  0.8701, Adjusted R-squared:  0.8659 
#> F-statistic: 204.3 on 2 and 61 DF,  p-value: < 2.2e-16
\end{verbatim}

When \(\gamma_1 = 0.25, \gamma_2 = 0\), we have the following
relationship:

\begin{Shaded}
\begin{Highlighting}[]
\NormalTok{idat }\OtherTok{\textless{}{-}} \FunctionTok{generate\_data}\NormalTok{(}\AttributeTok{effect =} \StringTok{"nonlinear"}\NormalTok{, }\AttributeTok{Zeffect =} \ConstantTok{TRUE}\NormalTok{, }\AttributeTok{Ueffect =} \ConstantTok{FALSE}\NormalTok{)}
\FunctionTok{summary\_scenario}\NormalTok{(idat)}
\end{Highlighting}
\end{Shaded}

\includegraphics{C:/Users/micsac/OneDrive - KI.SE/Methods/Bayesian DPP Surrogate/dpsurrogate/vignettes/scenarios_files/figure-latex/nonlin10-1.pdf}

\begin{verbatim}
#> 
#> Call:
#> lm(formula = yeff ~ seff + trtZ, data = idat$rdat)
#> 
#> Residuals:
#>     Min      1Q  Median      3Q     Max 
#> -1.4788 -0.3829  0.1867  0.3963  0.6035 
#> 
#> Coefficients:
#>             Estimate Std. Error t value Pr(>|t|)    
#> (Intercept) -0.31362    0.06844  -4.582 2.33e-05 ***
#> seff         0.97299    0.07022  13.857  < 2e-16 ***
#> trtZ         0.10742    0.06756   1.590    0.117    
#> ---
#> Signif. codes:  0 '***' 0.001 '**' 0.01 '*' 0.05 '.' 0.1 ' ' 1
#> 
#> Residual standard error: 0.513 on 61 degrees of freedom
#> Multiple R-squared:  0.7596, Adjusted R-squared:  0.7517 
#> F-statistic: 96.35 on 2 and 61 DF,  p-value: < 2.2e-16
\end{verbatim}

When \(\gamma_1 = 0, \gamma_2 = 0.25\), we have the following
relationship:

\begin{Shaded}
\begin{Highlighting}[]
\NormalTok{idat }\OtherTok{\textless{}{-}} \FunctionTok{generate\_data}\NormalTok{(}\AttributeTok{effect =} \StringTok{"nonlinear"}\NormalTok{, }\AttributeTok{Zeffect =} \ConstantTok{FALSE}\NormalTok{, }\AttributeTok{Ueffect =} \ConstantTok{TRUE}\NormalTok{)}
\FunctionTok{summary\_scenario}\NormalTok{(idat)}
\end{Highlighting}
\end{Shaded}

\includegraphics{C:/Users/micsac/OneDrive - KI.SE/Methods/Bayesian DPP Surrogate/dpsurrogate/vignettes/scenarios_files/figure-latex/nonlin01-1.pdf}

\begin{verbatim}
#> 
#> Call:
#> lm(formula = yeff ~ seff + trtZ, data = idat$rdat)
#> 
#> Residuals:
#>     Min      1Q  Median      3Q     Max 
#> -1.4528 -0.3335  0.0589  0.3234  1.1215 
#> 
#> Coefficients:
#>             Estimate Std. Error t value Pr(>|t|)    
#> (Intercept) -0.20673    0.06680  -3.095  0.00297 ** 
#> seff         1.18433    0.06425  18.433  < 2e-16 ***
#> trtZ         0.14746    0.05952   2.477  0.01602 *  
#> ---
#> Signif. codes:  0 '***' 0.001 '**' 0.01 '*' 0.05 '.' 0.1 ' ' 1
#> 
#> Residual standard error: 0.517 on 61 degrees of freedom
#> Multiple R-squared:  0.8596, Adjusted R-squared:  0.855 
#> F-statistic: 186.7 on 2 and 61 DF,  p-value: < 2.2e-16
\end{verbatim}

When \(\gamma_1 = \gamma_2 = 0.25\), we have the following relationship:

\begin{Shaded}
\begin{Highlighting}[]
\NormalTok{idat }\OtherTok{\textless{}{-}} \FunctionTok{generate\_data}\NormalTok{(}\AttributeTok{effect =} \StringTok{"nonlinear"}\NormalTok{, }\AttributeTok{Zeffect =} \ConstantTok{TRUE}\NormalTok{, }\AttributeTok{Ueffect =} \ConstantTok{TRUE}\NormalTok{)}
\FunctionTok{summary\_scenario}\NormalTok{(idat)}
\end{Highlighting}
\end{Shaded}

\includegraphics{C:/Users/micsac/OneDrive - KI.SE/Methods/Bayesian DPP Surrogate/dpsurrogate/vignettes/scenarios_files/figure-latex/nonlin11-1.pdf}

\begin{verbatim}
#> 
#> Call:
#> lm(formula = yeff ~ seff + trtZ, data = idat$rdat)
#> 
#> Residuals:
#>      Min       1Q   Median       3Q      Max 
#> -1.16829 -0.41350  0.01023  0.35952  1.00502 
#> 
#> Coefficients:
#>              Estimate Std. Error t value Pr(>|t|)    
#> (Intercept) -0.516091   0.068352  -7.551 2.61e-10 ***
#> seff         1.124208   0.060664  18.532  < 2e-16 ***
#> trtZ         0.007399   0.067747   0.109    0.913    
#> ---
#> Signif. codes:  0 '***' 0.001 '**' 0.01 '*' 0.05 '.' 0.1 ' ' 1
#> 
#> Residual standard error: 0.5431 on 61 degrees of freedom
#> Multiple R-squared:  0.8494, Adjusted R-squared:  0.8445 
#> F-statistic: 172.1 on 2 and 61 DF,  p-value: < 2.2e-16
\end{verbatim}

\hypertarget{linear}{%
\subsection{Linear}\label{linear}}

\[\mu_j = 1 - 1 \nu_j - \gamma_1 Z_j - \gamma_2 U_j.\]

When \(\gamma_1 = \gamma_2 = 0\), we have the following relationship:

\begin{Shaded}
\begin{Highlighting}[]
\NormalTok{idat }\OtherTok{\textless{}{-}} \FunctionTok{generate\_data}\NormalTok{(}\AttributeTok{effect =} \StringTok{"linear"}\NormalTok{, }\AttributeTok{Zeffect =} \ConstantTok{FALSE}\NormalTok{, }\AttributeTok{Ueffect =} \ConstantTok{FALSE}\NormalTok{)}
\FunctionTok{summary\_scenario}\NormalTok{(idat)}
\CommentTok{\#\textgreater{} Warning in summary.lm(lm(yeff \textasciitilde{} seff + trtZ, data = idat$rdat)): essentially}
\CommentTok{\#\textgreater{} perfect fit: summary may be unreliable}
\end{Highlighting}
\end{Shaded}

\includegraphics{C:/Users/micsac/OneDrive - KI.SE/Methods/Bayesian DPP Surrogate/dpsurrogate/vignettes/scenarios_files/figure-latex/lin00-1.pdf}

\begin{verbatim}
#> 
#> Call:
#> lm(formula = yeff ~ seff + trtZ, data = idat$rdat)
#> 
#> Residuals:
#>        Min         1Q     Median         3Q        Max 
#> -5.840e-16 -2.156e-16 -1.466e-16 -6.080e-17  9.246e-15 
#> 
#> Coefficients:
#>              Estimate Std. Error   t value Pr(>|t|)    
#> (Intercept) 1.000e+00  1.531e-16 6.533e+15   <2e-16 ***
#> seff        1.000e+00  1.411e-16 7.087e+15   <2e-16 ***
#> trtZ        2.524e-17  1.574e-16 1.600e-01    0.873    
#> ---
#> Signif. codes:  0 '***' 0.001 '**' 0.01 '*' 0.05 '.' 0.1 ' ' 1
#> 
#> Residual standard error: 1.201e-15 on 61 degrees of freedom
#> Multiple R-squared:      1,  Adjusted R-squared:      1 
#> F-statistic: 2.526e+31 on 2 and 61 DF,  p-value: < 2.2e-16
\end{verbatim}

When \(\gamma_1 = 0.25, \gamma_2 = 0\), we have the following
relationship:

\begin{Shaded}
\begin{Highlighting}[]
\NormalTok{idat }\OtherTok{\textless{}{-}} \FunctionTok{generate\_data}\NormalTok{(}\AttributeTok{effect =} \StringTok{"linear"}\NormalTok{, }\AttributeTok{Zeffect =} \ConstantTok{TRUE}\NormalTok{, }\AttributeTok{Ueffect =} \ConstantTok{FALSE}\NormalTok{)}
\FunctionTok{summary\_scenario}\NormalTok{(idat)}
\end{Highlighting}
\end{Shaded}

\includegraphics{C:/Users/micsac/OneDrive - KI.SE/Methods/Bayesian DPP Surrogate/dpsurrogate/vignettes/scenarios_files/figure-latex/lin10-1.pdf}

\begin{verbatim}
#> 
#> Call:
#> lm(formula = yeff ~ seff + trtZ, data = idat$rdat)
#> 
#> Residuals:
#>      Min       1Q   Median       3Q      Max 
#> -0.54450 -0.09026  0.00940  0.15649  0.25958 
#> 
#> Coefficients:
#>              Estimate Std. Error t value Pr(>|t|)    
#> (Intercept)  0.776945   0.022272  34.884   <2e-16 ***
#> seff         0.976576   0.021445  45.538   <2e-16 ***
#> trtZ        -0.002619   0.019924  -0.131    0.896    
#> ---
#> Signif. codes:  0 '***' 0.001 '**' 0.01 '*' 0.05 '.' 0.1 ' ' 1
#> 
#> Residual standard error: 0.1777 on 61 degrees of freedom
#> Multiple R-squared:  0.9723, Adjusted R-squared:  0.9714 
#> F-statistic:  1071 on 2 and 61 DF,  p-value: < 2.2e-16
\end{verbatim}

When \(\gamma_1 = 0, \gamma_2 = 0.25\), we have the following
relationship:

\begin{Shaded}
\begin{Highlighting}[]
\NormalTok{idat }\OtherTok{\textless{}{-}} \FunctionTok{generate\_data}\NormalTok{(}\AttributeTok{effect =} \StringTok{"linear"}\NormalTok{, }\AttributeTok{Zeffect =} \ConstantTok{FALSE}\NormalTok{, }\AttributeTok{Ueffect =} \ConstantTok{TRUE}\NormalTok{)}
\FunctionTok{summary\_scenario}\NormalTok{(idat)}
\end{Highlighting}
\end{Shaded}

\includegraphics{C:/Users/micsac/OneDrive - KI.SE/Methods/Bayesian DPP Surrogate/dpsurrogate/vignettes/scenarios_files/figure-latex/lin01-1.pdf}

\begin{verbatim}
#> 
#> Call:
#> lm(formula = yeff ~ seff + trtZ, data = idat$rdat)
#> 
#> Residuals:
#>      Min       1Q   Median       3Q      Max 
#> -0.37720 -0.16506 -0.01754  0.12497  0.66684 
#> 
#> Coefficients:
#>              Estimate Std. Error t value Pr(>|t|)    
#> (Intercept)  1.004017   0.029248  34.328   <2e-16 ***
#> seff         1.009724   0.030369  33.249   <2e-16 ***
#> trtZ        -0.002374   0.026656  -0.089    0.929    
#> ---
#> Signif. codes:  0 '***' 0.001 '**' 0.01 '*' 0.05 '.' 0.1 ' ' 1
#> 
#> Residual standard error: 0.2332 on 61 degrees of freedom
#> Multiple R-squared:  0.949,  Adjusted R-squared:  0.9474 
#> F-statistic:   568 on 2 and 61 DF,  p-value: < 2.2e-16
\end{verbatim}

When \(\gamma_1 = \gamma_2 = 0.25\), we have the following relationship:

\begin{Shaded}
\begin{Highlighting}[]
\NormalTok{idat }\OtherTok{\textless{}{-}} \FunctionTok{generate\_data}\NormalTok{(}\AttributeTok{effect =} \StringTok{"linear"}\NormalTok{, }\AttributeTok{Zeffect =} \ConstantTok{TRUE}\NormalTok{, }\AttributeTok{Ueffect =} \ConstantTok{TRUE}\NormalTok{)}
\FunctionTok{summary\_scenario}\NormalTok{(idat)}
\end{Highlighting}
\end{Shaded}

\includegraphics{C:/Users/micsac/OneDrive - KI.SE/Methods/Bayesian DPP Surrogate/dpsurrogate/vignettes/scenarios_files/figure-latex/lin11-1.pdf}

\begin{verbatim}
#> 
#> Call:
#> lm(formula = yeff ~ seff + trtZ, data = idat$rdat)
#> 
#> Residuals:
#>      Min       1Q   Median       3Q      Max 
#> -0.89928 -0.17864 -0.00527  0.18925  0.75112 
#> 
#> Coefficients:
#>             Estimate Std. Error t value Pr(>|t|)    
#> (Intercept)  0.82559    0.04505  18.328   <2e-16 ***
#> seff         0.97547    0.05054  19.301   <2e-16 ***
#> trtZ        -0.04663    0.04215  -1.106    0.273    
#> ---
#> Signif. codes:  0 '***' 0.001 '**' 0.01 '*' 0.05 '.' 0.1 ' ' 1
#> 
#> Residual standard error: 0.3489 on 61 degrees of freedom
#> Multiple R-squared:  0.8597, Adjusted R-squared:  0.8551 
#> F-statistic: 186.9 on 2 and 61 DF,  p-value: < 2.2e-16
\end{verbatim}

\hypertarget{null}{%
\subsection{Null}\label{null}}

\[\mu_j = 0 - 0 \nu_j - \gamma_1 Z_j - \gamma_2 U_j.\]

When \(\gamma_1 = \gamma_2 = 0\), we have the following relationship:

\begin{Shaded}
\begin{Highlighting}[]
\NormalTok{idat }\OtherTok{\textless{}{-}} \FunctionTok{generate\_data}\NormalTok{(}\AttributeTok{effect =} \StringTok{"null"}\NormalTok{, }\AttributeTok{Zeffect =} \ConstantTok{FALSE}\NormalTok{, }\AttributeTok{Ueffect =} \ConstantTok{FALSE}\NormalTok{)}
\FunctionTok{summary\_scenario}\NormalTok{(idat)}
\end{Highlighting}
\end{Shaded}

\includegraphics{C:/Users/micsac/OneDrive - KI.SE/Methods/Bayesian DPP Surrogate/dpsurrogate/vignettes/scenarios_files/figure-latex/null00-1.pdf}

\begin{verbatim}
#> 
#> Call:
#> lm(formula = yeff ~ seff + trtZ, data = idat$rdat)
#> 
#> Residuals:
#>    Min     1Q Median     3Q    Max 
#>      0      0      0      0      0 
#> 
#> Coefficients:
#>             Estimate Std. Error t value Pr(>|t|)
#> (Intercept)        0          0     NaN      NaN
#> seff               0          0     NaN      NaN
#> trtZ               0          0     NaN      NaN
#> 
#> Residual standard error: 0 on 61 degrees of freedom
#> Multiple R-squared:    NaN,  Adjusted R-squared:    NaN 
#> F-statistic:   NaN on 2 and 61 DF,  p-value: NA
\end{verbatim}

When \(\gamma_1 = 0.25, \gamma_2 = 0\), we have the following
relationship:

\begin{Shaded}
\begin{Highlighting}[]
\NormalTok{idat }\OtherTok{\textless{}{-}} \FunctionTok{generate\_data}\NormalTok{(}\AttributeTok{effect =} \StringTok{"null"}\NormalTok{, }\AttributeTok{Zeffect =} \ConstantTok{TRUE}\NormalTok{, }\AttributeTok{Ueffect =} \ConstantTok{FALSE}\NormalTok{)}
\FunctionTok{summary\_scenario}\NormalTok{(idat)}
\end{Highlighting}
\end{Shaded}

\includegraphics{C:/Users/micsac/OneDrive - KI.SE/Methods/Bayesian DPP Surrogate/dpsurrogate/vignettes/scenarios_files/figure-latex/null10-1.pdf}

\begin{verbatim}
#> 
#> Call:
#> lm(formula = yeff ~ seff + trtZ, data = idat$rdat)
#> 
#> Residuals:
#>      Min       1Q   Median       3Q      Max 
#> -0.38443 -0.07730  0.01868  0.12789  0.23381 
#> 
#> Coefficients:
#>             Estimate Std. Error t value Pr(>|t|)    
#> (Intercept) -0.22696    0.01970 -11.522   <2e-16 ***
#> seff        -0.01486    0.02528  -0.588    0.559    
#> trtZ         0.02162    0.01794   1.205    0.233    
#> ---
#> Signif. codes:  0 '***' 0.001 '**' 0.01 '*' 0.05 '.' 0.1 ' ' 1
#> 
#> Residual standard error: 0.1547 on 61 degrees of freedom
#> Multiple R-squared:  0.03308,    Adjusted R-squared:  0.00138 
#> F-statistic: 1.044 on 2 and 61 DF,  p-value: 0.3584
\end{verbatim}

When \(\gamma_1 = 0, \gamma_2 = 0.25\), we have the following
relationship:

\begin{Shaded}
\begin{Highlighting}[]
\NormalTok{idat }\OtherTok{\textless{}{-}} \FunctionTok{generate\_data}\NormalTok{(}\AttributeTok{effect =} \StringTok{"null"}\NormalTok{, }\AttributeTok{Zeffect =} \ConstantTok{FALSE}\NormalTok{, }\AttributeTok{Ueffect =} \ConstantTok{TRUE}\NormalTok{)}
\FunctionTok{summary\_scenario}\NormalTok{(idat)}
\end{Highlighting}
\end{Shaded}

\includegraphics{C:/Users/micsac/OneDrive - KI.SE/Methods/Bayesian DPP Surrogate/dpsurrogate/vignettes/scenarios_files/figure-latex/null01-1.pdf}

\begin{verbatim}
#> 
#> Call:
#> lm(formula = yeff ~ seff + trtZ, data = idat$rdat)
#> 
#> Residuals:
#>      Min       1Q   Median       3Q      Max 
#> -0.73529 -0.12776  0.05414  0.18675  0.68467 
#> 
#> Coefficients:
#>              Estimate Std. Error t value Pr(>|t|)
#> (Intercept)  0.016935   0.035066   0.483    0.631
#> seff        -0.009874   0.040836  -0.242    0.810
#> trtZ         0.025023   0.035519   0.704    0.484
#> 
#> Residual standard error: 0.2771 on 61 degrees of freedom
#> Multiple R-squared:  0.008716,   Adjusted R-squared:  -0.02379 
#> F-statistic: 0.2682 on 2 and 61 DF,  p-value: 0.7657
\end{verbatim}

When \(\gamma_1 = \gamma_2 = 0.25\), we have the following relationship:

\begin{Shaded}
\begin{Highlighting}[]
\NormalTok{idat }\OtherTok{\textless{}{-}} \FunctionTok{generate\_data}\NormalTok{(}\AttributeTok{effect =} \StringTok{"null"}\NormalTok{, }\AttributeTok{Zeffect =} \ConstantTok{TRUE}\NormalTok{, }\AttributeTok{Ueffect =} \ConstantTok{TRUE}\NormalTok{)}
\FunctionTok{summary\_scenario}\NormalTok{(idat)}
\end{Highlighting}
\end{Shaded}

\includegraphics{C:/Users/micsac/OneDrive - KI.SE/Methods/Bayesian DPP Surrogate/dpsurrogate/vignettes/scenarios_files/figure-latex/null11-1.pdf}

\begin{verbatim}
#> 
#> Call:
#> lm(formula = yeff ~ seff + trtZ, data = idat$rdat)
#> 
#> Residuals:
#>      Min       1Q   Median       3Q      Max 
#> -0.69535 -0.16619 -0.01165  0.17039  0.61288 
#> 
#> Coefficients:
#>             Estimate Std. Error t value Pr(>|t|)    
#> (Intercept) -0.19285    0.03706  -5.204 2.42e-06 ***
#> seff         0.06011    0.03423   1.756    0.084 .  
#> trtZ         0.01467    0.03052   0.481    0.633    
#> ---
#> Signif. codes:  0 '***' 0.001 '**' 0.01 '*' 0.05 '.' 0.1 ' ' 1
#> 
#> Residual standard error: 0.2879 on 61 degrees of freedom
#> Multiple R-squared:  0.04942,    Adjusted R-squared:  0.01825 
#> F-statistic: 1.586 on 2 and 61 DF,  p-value: 0.2132
\end{verbatim}

\end{document}
